
\section{Action of a group on a Set}
\subsection{Definition of an Action}
consider a group G and a set M. We will define the \textit{Action} of G on M as a mapping $G\times M \longrightarrow M $. the mapping is defined in such a way that it takes the pair $(a,m)$ to $am$ where $a \epsilon G$ and m and am $\epsilon M$. Which satisfies the associative law:
\begin{equation}
    a(bm)=(ab)m
\end{equation}
and also there there should be an identity element e which gives em=m for any m belonging to M. So, the action of a group on a set M is an homomorphism from the group G to the one-to-one transformation of M. An example could be the action of the group SL(2,$\mathbb{C}$) on the set of Minkowski space which is shown in the above section.

\subsection{Orbit of the point m under action of G}
The orbit of the point m is denoted by $G\cdot m$, this set is defined as:
\begin{equation}
    G\cdot m=\{x|x\epsilon M \; \text{such that} \; x =am \; \forall \; a\epsilon G , \; m\epsilon M \; \}
\end{equation}
That is, let G act on M ans let $m$ be a point in M, then the orbit of $m$ is defined as the subset of M in which the elements are of the form of all am for all a belonging to G and m belonging to M.

For example, Consider the action of the group SO(3) on the set of all points in the 3-Dimensional Euclidean space. We know SO(3) is the group of all the rotations in the space which conserves the length or the magnitude of the vector to that point. So by definition of the orbit, orbit of any point in the space except the origin is the collection of all points equidistant from the origin, that is its a sphere centered at the origin and the orbit of the origin is only the origin itself because the result of rotation of origin is the origin itself.


\subsection{Isotropy Group of point m}
The orbit group of point $m$ is a sub set of the $G$ denoted by $G_m$ which is defined as:
\begin{equation}
    G_m = \{ x | \; xm=m \; \forall \; m  \epsilon M\; \text{and} \; x\; \epsilon G \;\}
\end{equation}
Here, $G$ acts on $M$, and the isotropy group of $m$ is the set of all points in $G$ that preserves the point $m$. For example, the group SO(3) acts on the 3-D space. consider a point $ m\neq 0$. So the isotropy group of the point m is the set of all rotations which is about the point m. So, the $m$ is preserved. The isotropy group of the point $m=0$ that is the origin is the whole group of SO(3).

Let $\#G$ denote the number of elements in the group $G$. let $G$ act on $M$, consider a point $m$ in M, consider the orbit of $m$ under G. the number of elements in the orbit group of m is given by $ \# ( G\cdot m ) $. So, if $n$ is an element in $G\cdot m$, then $n=am$ for some a in G. If n=bm as well then $a^{-1}b$ must lie in $G_m$. So, this means that to every element n there are $\#G_m$ elements which map m to n.
So we can write:
\begin{equation}
    \#G = \# G \cdot m \times \# G_m
\end{equation}

\subsection{Cosets}

We say that G acts transitively on M if M has only a single orbit. The coset is denoted by $aG_{m}$  defined as:
\begin{equation}
   aG_m= \{ab \; | \; \forall \;b\; in\; G_m\}    
\end{equation}
Each coset contains the same number of elements as the subgroup $G_m$. The subgroup $Gm$'s number of elements is shared by all cosets. The point am may be identified with the coset $aG_m$, which is made up of all group elements that transport the point m into the point am. Since G operates transitively on M, there exists at least one element b of G such that n=bm for each element n of M. As a result, we have connected every point of M with a coset of G. We may identify the set of cosets by using the notation $G/G_m$ ~\cite{sternberg1995group}


