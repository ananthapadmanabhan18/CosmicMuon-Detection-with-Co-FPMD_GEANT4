\section{Introduction}
\subsection{Basic Definition}
In this section we will discuss about the basic definition of the groups. A group is usually denoted by the letter G. its none other than a set along with a defined operator usually called as group multiplication. The group multiplication can be any well defined operation like addition, subtraction etc.The operation is defined as a mapping from G$\times$ G $\longrightarrow$ G.That is this operation must satisfy the below four properties for a set to become a group.This group operation or multiplication is called as group axioms.
\begin{itemize}
    \item Closure Property-Is if a,b $\epsilon$ G and if ab represent the multiplication then ab must also belong to G. 
    \item Associativity- That is the group multiplication must follow
    \begin{equation}
        a(bc)=(ab)c
    \end{equation}
    Where a,b,c$\epsilon$G
    \item Existence of an Identity element-For a group there must be an identity element which satisfies 
    \begin{equation}
       ae=ea=a   
    \end{equation}
    
   for all a $\epsilon$ G
   
   \item Existence of an inverse element- this means that for every element"a" in G there must be an inverse element denoted by $a^{-1}$ which satisfies 
   \begin{equation}
       aa^{-1}=a^{-1}a=e
    \end{equation}
\end{itemize}

\subsection{Examples of groups}
\begin{itemize}
    \item Example:1- Consider the set G which is a set of 4 given 2$\times$2 matrices a,b,c and e and the operation of matrix multiplication. We can see that the set G is a group as it satisfies the above mentioned 4 conditions.
    \begin{center}
    a=
    $\begin{pmatrix}
    0 & -1 \\
    1 & 0 
    \end{pmatrix}$
      b=
     $\begin{pmatrix}
    -1 & 0 \\
    0 & -1 
    \end{pmatrix}$
    c=
    $\begin{pmatrix}
    0 & 1 \\
    -1 & 0 
    \end{pmatrix}$
    
    e=
    $\begin{pmatrix}
    1 & 0 \\
    0 & 1 
    \end{pmatrix}$
    
    Here we can see that
    
    ac=
    $\begin{pmatrix}
    0 & -1 \\
    1 & 0 
    \end{pmatrix}$
    $\begin{pmatrix}
    0 & 1 \\
    -1 & 0 
    \end{pmatrix}$
    =
    $\begin{pmatrix}
    1 & 0 \\
    0 & 1 
    \end{pmatrix}$
    =e
    
    $a^{-1}$=
    $\begin{pmatrix}
    0 & -1 \\
    1 & 0 
    \end{pmatrix}^{-1}$
    =
    c
    =
    $\begin{pmatrix}
    0 & 1 \\
    -1 & 0 
    \end{pmatrix}$
    \end{center}
    \item Example:2- Consider the set of all 2$\times$2 matrices with complex entries with determinant 1. We know that this is an infinite set. Let the operation be ordinary matrix multiplication. Consider the matrix A.
    \begin{center}
      A=
    $\begin{pmatrix}
    a & b \\
    c & d 
    \end{pmatrix}$  
    \end{center}
    
    So the condition is ad-bc=1.We will denote this as SL(2,C). This set is also a group as det(AB)=det(A)det(B) for all the matrices in the set So if A and B belong to the SL(2,C) then AB also belong to the same. We already know that matrix multiplication is Associative. and there is an identity element which is the usual 2x2 identity matrix and inverse of any matrix A is $A^{-1}$ which also belong to SL(2,C).
    \item Example:3- The set O(3) that is the set of all orthogonal transformation in the Euclidean 3-D space which preserve the Euclidean distance. This set is also a group as it satisfies all the properties of the group.
    
\end{itemize}

\subsection{Abelian and Non-Abelian Group}
An Abelian group is one in which the order of application of the group operation does not change the result. in other words for a group G, So for all a,b $\epsilon$ G, then if ab=ba,then the group G is an Abelian group. For example, the set of integers under addition operation is an Abelian group. An the groups which are not Abelian are called Non-Abelian groups. An example would be the rotation group SO(3) in three dimensions (for example, rotating something 90 degrees along one axis and then 90 degrees along a different axis is not the same as doing them in reverse order).



