
\section{Homomorphism} \label{Homomorphisms}




\subsection{Definition}
So far we have discussed about the definition of groups. Now lets look at Homomorphisms. Consider two groups $G_{1}$ and $G_{2}$. Consider a mapping $\phi$ from $G_{1}$ to $G_{2}$, We say that $\phi$ is a Homomorphism if:
\begin{equation}
    \phi(ab)=\phi(a)\phi(b)
\end{equation}
for all a and b $ \epsilon  G_{1}$

consider an example, consider the Homomorphism from the set $\mathbb{Z}$ to the set $C_{k}$ where the latter is the set of all $k^{th}$ root of unity. The function is given by:
\begin{equation}
    \phi(a)=e^{\frac{2\pi ia}{k}}=\omega^{a}
\end{equation}
We can see that this is a Homomorphism as $\phi(ab)=\phi(a)\phi(b)$ as $\omega^{a+b}=\omega^{a}\omega^{b}$















\subsection{Relation between SL(2,C) and the Lorentz group}
Now we will look at the relation between the SL(2,C) and the Lorentz group. We know that the Lorentz Group is the group of all the 4x4 matrices of Lorentz transformation of the Minkowski space. Minkowski space is the 4-D space (4x1 matrices)including t,x,y and z. Let M denote the 4 dimensional space M=$\mathbb{R}^{4}$. This is set up in such a way that the speed of light is unity. So the magnitude of the 4-D vector is given by:
\begin{equation}
    ||X||^2=x_{0}^{2}-x_{1}^{2}-x_{2}^{2}-x_{3}^{2} 
\end{equation}
 where  
\begin{equation}
X=
\begin{bmatrix}
x_0 \\
x_1 \\
x_2 \\
x_3 
\end{bmatrix}
\end{equation} 
Let L denote the Lorentz group. Consider the Lorentz transformation B which is the linear transformation of M into itself while the Lorentz metric or the magnitude remaining the same, that is $||BX||=||X||$ $\forall$ X $\epsilon$ M. 
Now we will describe a homomorphism from the group SL(2,$\mathbb{C}$). We will map every point in M (X) by a self adjoint 2x2 matrix x which is given by:
\begin{center}

X=
$\begin{pmatrix}
x_0 \\
x_1 \\
x_2 \\
x_3 
\end{pmatrix}$
$\longrightarrow$
$x$=
    $\begin{pmatrix}
    x_0+x_3 & x_1-ix_2 \\
    x_1+ix_2 & x_0-x_3
    \end{pmatrix}$
\end{center}
We can see that, by construction, the x is an Hermitian matrix, that is, its hermitian conjugate is itself. that is $x^{\dagger}=x$. We can also see that the square of the determinant of x is same as that of X ie, 
\begin{equation}
  ||X||^2=det(x)=x_{0}^{2}-x_{1}^{2}-x_{2}^{2}-x_{3}^{2}   
\end{equation}
We know that the matrix $x$ is a self adjoint matrix,  so $x^*=x$. And obviously we can write all of these kind of matrices as a linear combination of the Pauli matrices which are given as:
\begin{center}
    $\sigma_1=$
    $\begin{pmatrix}
    0 & 1 \\
    1 & 0 
    \end{pmatrix}$
    $\sigma_2=$
     $\begin{pmatrix}
    0 & -i \\
    i & 0 
    \end{pmatrix}$
    $\sigma_3=$
    $\begin{pmatrix}
    1 & 0 \\
    0 & -1 
    \end{pmatrix}$
    
    e=
    $\begin{pmatrix}
    1 & 0 \\
    0 & 1 
    \end{pmatrix}$  (The Identity Matrix)
    \end{center}
    
So our defined matrix X can be represented as $x$ which in turn can be represented in terms of Pauli matrices as:
\begin{equation}
    x=x_0e+x_1\sigma_1+x_2\sigma_2+x_3\sigma_3
\end{equation}


Now let us define an actionof the matrix $A$ on the self adjoint matrix $x$ denoted by $\phi (A)x$ and given by:
\begin{equation}
    x \longrightarrow AxA^{*}
\end{equation}
We define the Corresponding Action on the vector $X$ by:
\begin{equation}
    \phi(A)X=AxA
\end{equation}

We can see that since $x$ is a self adjoint matrix, $AxA^*$ is also self-adjoint. Also we can see that the determinants:
\begin{equation}
    det(AxA^{*})=|detA|^{2} \times det(x)
\end{equation}


So, if A is in SL(2,$\mathbb{C}$), then:
\begin{equation}
    ||\phi(A)X||^{2}=||X||^{2}
\end{equation}

and SL(2,$\mathbb{C}$) represents a Lorentz transformation. So we obtain:
\begin{equation}
    \phi(AB)X=\phi(A)\phi(B)X
\end{equation}


So we can prove that $\phi$ is a homomorphism.

Now, let A belong to the subgroup SU(2) of SO(3). So, A is also a unitary matrix. Let $e_0$ be the following matrix:
\begin{center}
    $e_0 =$
    $\begin{pmatrix}
    1  \\
    0  \\
    0  \\
    0  \\
    \end{pmatrix}$
\end{center}  

We can see that:

\begin{equation}
    \phi(A)e_0=e_0
\end{equation}
Since $e_0$ is represented by the identity matrix $I$. Also while restricting the $\phi$ to $SU(2)$, the mapping changes to $SU(2)$ to $O(3)$.

For exapmple. consider the matrix:


\begin{center}
\begin{equation}
U_{\theta}=
{
\begin{pmatrix}
  e^{-i\theta} & 0\\
  0 & e^{i\theta}
\end{pmatrix}
}  
\end{equation}
\end{center}

Now we have:
\begin{center}
\begin{equation}
    \phi({U_{\theta}})X=U_{\theta}xU_{-\theta}=
\end{equation}
\end{center}
\begin{center}
\begin{equation}
=
{
\begin{pmatrix}
  e^{-i\theta} & 0\\
  0 & e^{i\theta}
\end{pmatrix}
} 
{
    \begin{pmatrix}
    x_0+x_3 & x_1-ix_2 \\
    x_1+ix_2 & x_0-x_3
    \end{pmatrix}
}
{
\begin{pmatrix}
    e^{i\theta} & 0\\
  0 & e^{-i\theta}
\end{pmatrix}
}
\end{equation}
\end{center}

We get:
\begin{equation}
 \phi({U_{\theta}})X=
     \begin{pmatrix}
    x_0+x_3 & e^{-i2\theta}(x_1-ix_2) \\
    e^{i2\theta}(x_1+ix_2) & x_0-x_3
    \end{pmatrix}
\end{equation}

We can see that $\phi({U_{\theta}})$ is a rotation about $ x_3 $ axis in the $x_1 - x_2$ plane an Angle of $2\theta$. And as $\theta$ ranges from 0 to $\pi$ the rotation goes from 0 to $2\pi$. 

Similarly, consider the action of the matrix:
\begin{equation}
 V_{\alpha}=
     \begin{pmatrix}
    cos\alpha & -sin\alpha\\
    sin\alpha & cos\alpha
    \end{pmatrix}
\end{equation}

Similarly for the vectors $e_2$ and $e_3$:
\begin{equation}
e_2=
    \begin{pmatrix}
      0\\
      0\\
      1\\
      0\\
    \end{pmatrix}
    \text{and}\;
    e_3=
    \begin{pmatrix}
    0\\
    0\\
    0\\
    1\\
    \end{pmatrix}
\end{equation}

calculating $\phi(V_{\alpha})e_2$ and $\phi(V_{\alpha})e_3$ by the definition of the $\phi$ we get:
\begin{equation}
    \phi(V_{\alpha})e_2=e_2
\end{equation}
So, $\phi(V_{\alpha}$ is the rotation about the $x_2$ axis. and:
\begin{equation}
    \phi(V_{\alpha})e_3=
    {
    \begin{pmatrix}
    cos2\alpha & sin2\alpha\\
    sin2\alpha & -cos2\alpha
    \end{pmatrix}
    }
\end{equation}

This corresponds to the vector:
$
\begin{pmatrix}
0\\
sin2\alpha\\
0\\
cos2\alpha\\
\end{pmatrix}
$
Hence we can see that :
\begin{equation}
   \phi(V_{\alpha})e_3=e_3 cos2\alpha + e_1 sin2\alpha 
\end{equation}
So, we can conclude that the $V_{\alpha}$ is the rotation about the $x_2$ axis by an angle $2\theta$.





\subsection{Representation of Lorentz boost}
Now, Consider the matrix $M_r$ with Real entries Such that $M_r={M_r}^{*}$:
\begin{equation}
    M_{r}=
    {
    \begin{pmatrix}
    r&0\\
    0&\frac{1}{r}
    \end{pmatrix}
    }
\end{equation}

and calculating the $\phi(M_{r})X$ we get:
\begin{equation}
    \phi(M_{r})X=
    \begin{pmatrix}
      r^{2}(x_0 + x_3) & x_1 +ix_2\\
      x_1 -ix_2 & r^{-2}(x_0 - x_3)\\
    \end{pmatrix}
\end{equation}

Here we can see that the $M_r$ doesn't make any change to $x_1 \text{and} x_2$ but the $x_0 \text{and} x_3$ changes as:
\begin{equation}
    x_0' = \frac{r^2 + r^{-2}}{2}x_0 + \frac{r^2 - r^{-2}}{2}x_3
\end{equation}
and
\begin{equation}
    x_3' = \frac{r^2 - r^{-2}}{2}x_0 + \frac{r^2 + r^{-2}}{2}x_3
\end{equation}
Now, the Lorentz boost  in the z direction with parameter t is denoted by $L_t$ and is defined as the transformation:
\begin{equation}
    x_0' = cosh(t)x_0 + sinh(t)x_3
\end{equation}
and
\begin{equation}
    x_3' = sinh(t)x_0 + cosh(t)x_3
\end{equation}
or in matrix form the ${L_t}^z$ is given by:

\begin{equation}
{L_t}^z=
    \begin{pmatrix}
      cosh\;t &0&0&sinh\;t\\
      0&1&0&0\\
      0&0&1&0\\
      sinh\;t &0&0&cosh\;t\\
    \end{pmatrix}
\end{equation}

So, in the preceding section, if we put $r=exp(t)$ in equation 25 and doing the calculation, we can see that :
\begin{equation}
    \phi(M_{e^t})={L_t}^z
\end{equation}
So, till now we have shown that:
\begin{equation}
    \phi(U_{\theta})={R_{\theta}}^z
\end{equation}
\begin{equation}
    \phi(V_{\theta})={R_{\theta}}^y
\end{equation}
\begin{equation}
    \phi(M_{e^t})={L_{2t}}^z
\end{equation}
where ${R_{\theta}}^z \text{and} {R_{\theta}}^y$ are the rotation about z and y axis by an angle $\theta$.





\subsection{Proper Lorentz transformation}


\subsubsection{Definition}
The proper Lorentz transformation is denoted by $L^0$. It is the sub group of $L$ consisting of those transformations which have positive determinant and preserves the foreword light cone, that is, which send each component of the set of time like vectors into itself.


\subsubsection{Lemma}
 The lemma is that, every proper Lorentz transformation,$B$ can be written as:
 \begin{equation}
     B=R_1{L_u}^zR_2
 \end{equation}

Where $R_1$ and $R_2$ are rotations and  ${L_u}^zR_2$ is a lorentz boost in the z-direction.

\subsubsection{Proof:}
By definition we can write:
\begin{equation}
    Be_0=
    {
    \begin{pmatrix}
      x_0\\
      \textbf{x}\\
    \end{pmatrix}
    }
\end{equation}
Where \textbf{x}=$x_1e_1+x_2e_2+x_3e_3$ and $x_0^2-||x||^2=1$. We can find a rotation S in which rotates the vector x to z-axis. So:
\begin{equation}
    SBe_0=
    {
    \begin{pmatrix}
      x_0\\
      0\\
      0\\
      ||x||
    \end{pmatrix}
    }
\end{equation}
the corresponding self-adjoint matrix is:
\begin{center}
$\begin{pmatrix}
  x_0+||\textbf{x}||&0\\
  0&x_0-||\textbf{x}||&0\\
\end{pmatrix}$
\end{center}

Now by choosing r such that:
\begin{equation}
    r^2=\frac{1}{x_0+||\textbf{x}||}
\end{equation}
Then applying $M_r$ gives $\phi(M_r)SBe_0=e_0$. Thus $\phi(M_r)SB=R_2 \;(let)$ is a rotation and we have:
\begin{equation}
    \phi(M_r)SB=R_2
\end{equation}
or we have:
\begin{equation}
    B=S^{-1}[\phi(M_r)]^{-1}] R_1
\end{equation}
With $R_1=S^{-1}$ and $L_{u}^{z}=\phi(M_r)^{-1}$.
So, we have proved the lemma.