\documentclass{article}


\usepackage[utf8]{inputenc}
% \usepackage[utf8]{inputenc}
\usepackage{multicol}
\usepackage[a4paper,top=3cm,bottom=3cm,left=2cm,right=2cm,marginparwidth=1.75cm]{geometry}
\usepackage{multicol}
\usepackage{amsmath}
\usepackage{graphicx}
\usepackage{hyperref}
\hypersetup{colorlinks=true,linkcolor=black,filecolor=magenta,urlcolor=cyan,}
\usepackage{amsfonts}
\usepackage{mathtools}
\usepackage{lipsum}
\usepackage{float}


\begin{document}



%TitlePage%TitlePage%TitlePage%TitlePage%TitlePage%TitlePage%TitlePage%TitlePage%TitlePage%TitlePage%TitlePage%TitlePage%TitlePage
\newgeometry{top=4cm,bottom=2.5cm,left=4cm,right=4cm}
\begin{titlepage}

    \begin{center}
    
    
    
    \textup{\Large {\bf Summer Internship Project} \\ Report}\\[0.3in]
    
    % Title
    \Large \textbf {Detection of Cosmic Muons Using Gas Detectors}\\[0.7in]
    
    
           
    
    % Submitted by
    \normalsize Submitted by \\[0.2in]
    \textbf{Anantha Padmanabhan M Nair}\\
    \normalsize
    $4^{th}$ year Int. MSc Student\\[0.1in]
    
    \includegraphics[width=0.25 \textwidth]{NISER.png}\\[0.1in]
    \Large{School of Physical Sciences}\\
    \normalsize
    \textsc{National Institute of Science Education and Research},\\
    Tehsildar Office, Khurda\\
    Pipli, Near, Jatni, Odisha 752050\\
    
    
    
    
    
    \vspace{.2in}
    Under the guidance of\\[0.2in]
    \textbf{Dr. Sanjib Muhuri}\\
    Scientific Officer
    
    
    
    \vspace{.3in}
    
    % Bottom of the page
    \includegraphics[width=0.2\textwidth]{VECC.png}\\[0.1in]
    \Large{Experimental High Energy Physics}\\
    \normalsize
    \textsc{Variable Energy Cyclotron Centre}\\
    1/AF, \\Bidhannagar, Kolkata, \\West Bengal, 700064 \\
    \vspace{0.2in}
    Summer Internship 2023
    
    \end{center}
    
\end{titlepage}
\restoregeometry
%TitlePage%TitlePage%TitlePage%TitlePage%TitlePage%TitlePage%TitlePage%TitlePage%TitlePage%TitlePage%TitlePage%TitlePage%TitlePage%









%Acknowlegdements%Acknowlegdements%Acknowlegdements%Acknowlegdements%Acknowlegdements%Acknowlegdements%Acknowlegdements%Acknowlegdements
\newgeometry{top=4.5cm,bottom=3.5cm,left=3cm,right=3cm}
\cleardoublepage
\begin{center}
    \Large{\textbf{Acknowledgements}}
\end{center}

\vspace{0.2in}
I would like to express my sincere gratitude to Dr. Sanjib Muhuri, 
my guide and mentor, for his invaluable support, guidance, 
and expertise throughout my internship on the simulation and 
experimental detection of cosmic muons using the ALICE detector. 
His vast knowledge and constant encouragement have been instrumental 
in shaping my understanding of this complex field of study.
I am also grateful to the High Energy Group at VECC (Variable 
Energy Cyclotron Centre) for providing me with the necessary 
resources, infrastructure, and access to the ALICE detector. Their 
assistance and cooperation have been vital in carrying out the 
experimental phase of this internship.
I extend my heartfelt appreciation to the Department of School of 
Physical Sciences at NISER (National Institute of Science Education 
and Research) for providing me with the opportunity to pursue this 
internship. The conducive academic environment and the support of the 
faculty have greatly contributed to my overall learning experience.
I would like to acknowledge the collective efforts of the researchers, 
technicians, and staff members who have been associated with the 
project. Their contributions, suggestions, and collaboration have 
significantly enriched my internship journey.
Lastly, I express my gratitude to my fellow interns and 
friends for their camaraderie, stimulating discussions, and 
continuous encouragement. Their presence has made this internship 
an enjoyable and enriching experience.
In conclusion, I am immensely thankful to all individuals 
and institutions mentioned above for their unwavering support, 
guidance, and contributions, which have been instrumental in the 
successful completion of this internship.
\newpage
\restoregeometry
%Acknowlegdements%Acknowlegdements%Acknowlegdements%Acknowlegdements%Acknowlegdements%Acknowlegdements%Acknowlegdements%Acknowlegdements











%Certificate%Certificate%Certificate%Certificate%Certificate%Certificate%Certificate%Certificate%Certificate%Certificate%Certificate%Certificate
\newgeometry{top=4.5cm,bottom=3.5cm,left=4cm,right=4cm}
\newpage
\thispagestyle{empty}

\begin{center}

\huge{\textsc{Variable Energy Cyclotron Centre, Kolkata}}\\[1cm]
\normalsize

\emph{\LARGE Certificate}\\[1cm]
\end{center}



This is to certify that Mr. Anantha Padmanabhan M Nair, a 
student of the National Institute of Science Education and 
Research (NISER), has successfully completed a summer internship 
in the field of Cosmic Muons and its Detection by ALICE Detectors. 
The internship was conducted from 5/6/2023 to 29/7/2023, under the 
guidance and supervision of Dr. Sanjib Muhuri.

During the internship, Mr. M Nair demonstrated exceptional 
dedication, enthusiasm, and competence in conducting simulations, 
experimental data collection, and analysis related to cosmic muons 
and their detection using the ALICE detector. Through their hard 
work and perseverance, they contributed significantly to the 
understanding of cosmic muons' behavior and their energy deposition 
within the detector setup.

We commend Mr. M Nair for their exemplary performance, commitment 
to scientific inquiry, and collaborative spirit. 
Their active participation and insightful contributions have been 
instrumental in the success of the internship project.

We extend our best wishes to Mr. M Nair for a bright and 
successful future, both in their personal life and professional 
career. May they continue to excel in their academic pursuits and 
make significant contributions to the field of research.


% This is to certify that Mr. Anantha Padmanabhan, Student of 
% National Institute of Science Education and Research, has 
% successfully completed a summer internship in the field of 
% Cosmic Muons and its Detection by ALICE Detectors from 5/06/2023 to 29/07/2023 under the guidance 
% of Dr.Sanjib Muhuri. We wish him every success 
% in his life and career.\\
\vspace{0.5cm}
\begin{flushleft}
    VECC Kolkata,\\
    Bidhannagar,\\
    West Bengal\\
\end{flushleft}




\vfill


% Bottom of the page
\begin{flushright}
Sanjib Muhuri\\
(Project Guide)\\
\end{flushright}
\begin{flushleft}
Date:
\end{flushleft}
\restoregeometry
%Certificate%Certificate%Certificate%Certificate%Certificate%Certificate%Certificate%Certificate%Certificate%Certificate%Certificate












%Abstract%Abstract%Abstract%Abstract%Abstract%Abstract%Abstract%Abstract%Abstract%Abstract%Abstract%Abstract%Abstract%Abstract
\newgeometry{top=5cm,bottom=2.5cm,left=3cm,right=3cm}

\begin{abstract}
    This report focuses on the simulation and 
    experimental detection of cosmic muons using the ALICE 
    (A Large Ion Collider Experiment) detector, employing the 
    Geant4 simulation framework. The study aims to understand 
    the behavior of cosmic muons and their energy deposition 
    within the detector setup, consisting of a honeycomb gas 
    detector filled with Argon and $CO_2$. The simulation phase involved utilizing Geant4 to replicate 
    the laboratory setup and simulate the interaction of cosmic 
    muons with the detector. By accurately modeling the trajectory 
    and behavior of cosmic muons, the simulation facilitated the 
    calculation of their energy deposition within the ALICE detector. 
    This process provided valuable insights into the expected behavior 
    of cosmic muons within the simulated environment. 
    Subsequently, the actual experimental phase involved 
    collecting data from the ALICE detector in the laboratory. 
    The detector, filled with Argon and CO2 gases, accurately 
    captured cosmic muons as they traversed through it. The collected 
    data allowed for a comparison between the simulated and 
    experimental results, validating the accuracy of the Geant4 
    simulation model and providing further insights into the behavior 
    of cosmic muons.
    The internship report outlines the methodology employed during 
    both the Geant4 simulation and the experimental data collection 
    phases, including the parameters and variables considered. It 
    discusses the data analysis techniques used to evaluate the energy 
    deposition of cosmic muons and presents a comprehensive comparison 
    between the simulated and experimental results.
    
\end{abstract}
\restoregeometry
%Abstract%Abstract%Abstract%Abstract%Abstract%Abstract%Abstract%Abstract%Abstract%Abstract%Abstract%Abstract%Abstract%Abstract











%Content Table Page%Content Table Page%Content Table Page%Content Table Page%Content Table Page%Content Table Page
\pagenumbering{roman}
\newpage
\newgeometry{top=4cm,bottom=2.5cm,left=4cm,right=4cm}
\begin{center}
 \tableofcontents   
\end{center}
\restoregeometry
%Content Table Page%Content Table Page%Content Table Page%Content Table Page%Content Table Page%Content Table Page


\newpage
\begin{multicols}{2}
\pagenumbering{arabic}



%introduction%introduction%introduction%introduction%introduction%introduction%introduction%introduction%introduction%introduction
\section{Introduction}

hey-\cite{sternberg1995group}

%introduction%introduction%introduction%introduction%introduction%introduction%introduction%introduction%introduction%introduction





















\end{multicols}




\bibliographystyle{plain}
\bibliography{bib.bib}


\end{document}