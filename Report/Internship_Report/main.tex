\documentclass{article}


\usepackage[utf8]{inputenc}
% \usepackage[utf8]{inputenc}
\usepackage{multicol}
\usepackage{dcolumn}
\usepackage[a4paper,top=3cm,bottom=3cm,left=2cm,right=2cm,marginparwidth=1.75cm]{geometry}
\usepackage{multicol}
\usepackage{multirow}
\usepackage{amsmath}
\usepackage{graphicx}
\usepackage{hyperref}
\hypersetup{colorlinks=true,linkcolor=black,filecolor=magenta,urlcolor=cyan,}
\usepackage{amsfonts}
\usepackage{mathtools}
\usepackage{lipsum}
\usepackage{float}
\usepackage{layout}
\usepackage{bm}

\begin{document}



%TitlePage%TitlePage%TitlePage%TitlePage%TitlePage%TitlePage%TitlePage%TitlePage%TitlePage%TitlePage%TitlePage%TitlePage%TitlePage
\newgeometry{top=4cm,bottom=2.5cm,left=4cm,right=4cm}
\begin{titlepage}

    \begin{center}
    
    
    
    \textup{\Large {\bf Summer Internship Project} \\ Report}\\[0.3in]
    
    % Title
    \Large \textbf {Detection of Cosmic Muons Using Gas Detectors}\\[0.7in]
    
    
           
    
    % Submitted by
    \normalsize Submitted by \\[0.2in]
    \textbf{Anantha Padmanabhan M Nair}\\
    \normalsize
    $4^{th}$ year Int. MSc Student\\[0.1in]
    
    \includegraphics[width=0.25 \textwidth]{NISER.png}\\[0.1in]
    \Large{School of Physical Sciences}\\
    \normalsize
    \textsc{National Institute of Science Education and Research},\\
    Tehsildar Office, Khurda\\
    Pipli, Near, Jatni, Odisha 752050\\
    
    
    
    
    
    \vspace{.2in}
    Under the guidance of\\[0.2in]
    \textbf{Dr. Sanjib Muhuri}\\
    Scientific Officer
    
    
    
    \vspace{.3in}
    
    % Bottom of the page
    \includegraphics[width=0.2\textwidth]{VECC.png}\\[0.1in]
    \Large{Experimental High Energy Physics}\\
    \normalsize
    \textsc{Variable Energy Cyclotron Centre}\\
    1/AF, \\Bidhannagar, Kolkata, \\West Bengal, 700064 \\
    \vspace{0.2in}
    Summer Internship 2023
    
    \end{center}
    
\end{titlepage}
\restoregeometry
%TitlePage%TitlePage%TitlePage%TitlePage%TitlePage%TitlePage%TitlePage%TitlePage%TitlePage%TitlePage%TitlePage%TitlePage%TitlePage%









%Acknowledgements%Acknowledgements%Acknowledgements%Acknowledgements%Acknowledgements%Acknowledgements%Acknowledgements%Acknowledgements
\newgeometry{top=4.5cm,bottom=3.5cm,left=3cm,right=3cm}
\cleardoublepage
\begin{center}
    \Large{\textbf{Acknowledgements}}
\end{center}

\vspace{0.2in}
I would like to express my sincere gratitude to Dr. Sanjib Muhuri, 
my guide and mentor, for his invaluable support, guidance, 
and expertise throughout my internship on the simulation and 
experimental detection of cosmic muons using the ALICE detector. 
His vast knowledge and constant encouragement have been instrumental 
in shaping my understanding of this complex field of study.
I am also grateful to the High Energy Group at VECC (Variable 
Energy Cyclotron Centre) for providing me with the necessary 
resources, infrastructure, and access to the ALICE detector. Their 
assistance and cooperation have been vital in carrying out the 
experimental phase of this internship.
I extend my heartfelt appreciation to the Department of School of 
Physical Sciences at NISER (National Institute of Science Education 
and Research) for providing me with the opportunity to pursue this 
internship. The conducive academic environment and the support of the 
faculty have greatly contributed to my overall learning experience.
I would like to acknowledge the collective efforts of the researchers, 
technicians, and staff members who have been associated with the 
project. Their contributions, suggestions, and collaboration have 
significantly enriched my internship journey.
Lastly, I express my gratitude to my fellow interns and 
friends for their camaraderie, stimulating discussions, and 
continuous encouragement. Their presence has made this internship 
an enjoyable and enriching experience.
In conclusion, I am immensely thankful to all individuals 
and institutions mentioned above for their unwavering support, 
guidance, and contributions, which have been instrumental in the 
successful completion of this internship.
\newpage
\restoregeometry
%Acknowledgements%Acknowledgements%Acknowledgements%Acknowledgements%Acknowledgements%Acknowledgements%Acknowledgements%Acknowledgements











%Certificate%Certificate%Certificate%Certificate%Certificate%Certificate%Certificate%Certificate%Certificate%Certificate%Certificate%Certificate
\newgeometry{top=4.5cm,bottom=3.5cm,left=4cm,right=4cm}
\newpage
\thispagestyle{empty}

\begin{center}

\huge{\textsc{Variable Energy Cyclotron Centre, Kolkata}}\\[1cm]
\normalsize

\emph{\LARGE Certificate}\\[1cm]
\end{center}



This is to certify that Mr. Anantha Padmanabhan M Nair, a 
student of the National Institute of Science Education and 
Research (NISER), has successfully completed a summer internship 
in the field of Cosmic Muons and its Detection by ALICE Detectors. 
The internship was conducted from 5/6/2023 to 29/7/2023, under the 
guidance and supervision of Dr. Sanjib Muhuri.

During the internship, Mr. M Nair demonstrated exceptional 
dedication, enthusiasm, and competence in conducting simulations, 
experimental data collection, and analysis related to cosmic muons 
and their detection using the ALICE detector. Through their hard 
work and perseverance, they contributed significantly to the 
understanding of cosmic muons' behavior and their energy deposition 
within the detector setup.

We commend Mr. M Nair for their exemplary performance, commitment 
to scientific inquiry, and collaborative spirit. 
Their active participation and insightful contributions have been 
instrumental in the success of the internship project.

We extend our best wishes to Mr. M Nair for a bright and 
successful future, both in their personal life and professional 
career. May they continue to excel in their academic pursuits and 
make significant contributions to the field of research.


% This is to certify that Mr. Anantha Padmanabhan, Student of 
% National Institute of Science Education and Research, has 
% successfully completed a summer internship in the field of 
% Cosmic Muons and its Detection by ALICE Detectors from 5/06/2023 to 29/07/2023 under the guidance 
% of Dr.Sanjib Muhuri. We wish him every success 
% in his life and career.\\
\vspace{0.5cm}
\begin{flushleft}
    VECC Kolkata,\\
    Bidhannagar,\\
    West Bengal\\
\end{flushleft}




\vfill


% Bottom of the page
\begin{flushright}
Dr Sanjib Muhuri\\
(Project Guide)\\
\end{flushright}
\begin{flushleft}
Date:
\end{flushleft}
\restoregeometry
%Certificate%Certificate%Certificate%Certificate%Certificate%Certificate%Certificate%Certificate%Certificate%Certificate%Certificate












%Abstract%Abstract%Abstract%Abstract%Abstract%Abstract%Abstract%Abstract%Abstract%Abstract%Abstract%Abstract%Abstract%Abstract
\newgeometry{top=5cm,bottom=2.5cm,left=3cm,right=3cm}

\begin{abstract}
    This internship report presents a comprehensive study on the detection of 
    cosmic muons using the CoFPMD (Cosmic Flux Photon Multiplicity Detector). 
    The primary objectives were to construct the detector setup, investigate 
    the energy deposit by muons at various incidence energies, and simulate 
    the actual data from EcoMug to improve the detector's performance.
    The internship commenced with the construction of the CoFPMD detector, 
    which consists of a honeycomb gas detector filled with Argon and CO2. 
    Geant4, a powerful simulation toolkit, was utilized to replicate the 
    detector's geometry and model the interaction of cosmic muons with the 
    detector materials. The fParticleGun was employed to simulate the 
    firing of muons with different incidence energies, enabling the 
    assessment of their energy deposit patterns within the detector.
    The first phase of the internship involved analyzing the energy deposit by 
    muons for different incidence energies. The obtained results provided valuable 
    insights into the behavior of cosmic muons and their interaction with the Argon 
    and CO2 gas mixture. The data facilitated the understanding of the detector's 
    response to muon incidence at various energy levels.
    In the second phase, data from Simulation in EcoMug, containing momentum, 
    position, and energy information, was implemented in the Geant4 fParticleGun. 
    This implementation allowed for the generation of realistic muon events, 
    which closely represented the characteristics of actual cosmic muons. By 
    incorporating real data, the accuracy of the simulation was enhanced, 
    resulting in a more reliable representation of the detector's performance.
    Based on the simulated data, a plot of the energy deposit by muons 
    was generated. This plot not only demonstrated the sensitivity and 
    efficiency of the CoFPMD detector in capturing cosmic muons but also 
    highlighted the correlation between energy deposition and muon incidence energies.
    In conclusion, this internship has provided valuable hands-on experience in the field of cosmic muon detection and its simulation using Geant4. The results obtained through this study offer a solid foundation for future research and advancements in the realm of cosmic muon detection and particle physics.
    
\end{abstract}
\restoregeometry
%Abstract%Abstract%Abstract%Abstract%Abstract%Abstract%Abstract%Abstract%Abstract%Abstract%Abstract%Abstract%Abstract%Abstract











%Content Table Page%Content Table Page%Content Table Page%Content Table Page%Content Table Page%Content Table Page
\pagenumbering{roman}
\newpage
\newgeometry{top=4cm,bottom=2.5cm,left=4cm,right=4cm}
\begin{center}
 \tableofcontents   
\end{center}
\restoregeometry
%Content Table Page%Content Table Page%Content Table Page%Content Table Page%Content Table Page%Content Table Page


\newpage
\begin{multicols}{2}
\pagenumbering{arabic}



%introduction%introduction%introduction%introduction%introduction%introduction%introduction%introduction%introduction%introduction
\section{Introduction}

The detection and study of cosmic muons hold great significance in the field of particle physics 
and high-energy physics. Cosmic muons, which are highly energetic charged particles originating from 
cosmic rays, provide valuable insights into the properties of elementary particles and the fundamental 
forces governing our universe. To explore the behavior of cosmic muons and their interaction with matter, 
we are using the CoFPMD (Cosmic flux Photon Multiplicity Detectors) detectors. 


The internship comprises two main phases: simulation and experimentation. In the 
simulation phase, the Geant4 framework is utilized to model the laboratory setup 
and replicate the interaction of cosmic muons with the CoFPMD. Geant4, a 
widely used toolkit in high-energy physics, provides a comprehensive platform for 
simulating the passage of particles through matter, accurately capturing their interactions 
and energy deposition.


In this report, we will outline the methodology employed during the simulation and 
experimental phases, discuss the data analysis techniques used to evaluate the energy 
deposition of cosmic muons, present the comparison between simulated and experimental 
results, and provide a comprehensive analysis of the overall internship experience.


%introduction%introduction%introduction%introduction%introduction%introduction%introduction%introduction%introduction%introduction


%Cosmic Muons and its Properties%Cosmic Muons and its Properties%Cosmic Muons and its Properties%Cosmic Muons and its Properties%Cosmic Muons and its Properties



\section{Cosmic Muons and its Properties}



Cosmic rays are high-energy particles that originate from various sources beyond our 
solar system, such as distant stars, supernovae, and active galactic nuclei. They consist of 
protons(87\%),Alpha Particles(12\%), and atomic Heavy Nuclei (1\%), some of which can have energies millions or even billions 
of times greater than those produced in the most powerful particle accelerators on Earth. When 
cosmic rays enter the Earth's atmosphere, they interact with air molecules, producing a cascade 
of secondary particles, including muons, neutrinos, and gamma rays.

\subsection{Production of Cosmic Muons}\label{muonproduction}
When there Cosmic Rays Reach the Earths atmosphere, it collides with the air molecules and produces 
many particles, mostly Pions and Kaons. These are the primary Particles. These particles then decay to produce a wide variety
of particles most of which are muons. Also, more than 90\% of the cosmic muons are produced from Pions.

From the Kaons, We can see that the muons are produced by the weak interactions and it also produces neutrinos given by:
\begin{equation}
    K^{+} \longrightarrow \mu^{+} + \nu_{\mu}
\end{equation}

From the decay of Pions, 64\% of the time, disintegrates directly into $\mu^+$; 21\% of the time into $\pi^0$and$\pi^+$ and Only 6\% of the disintegrations produce three particles-$\mu^+$,$\mu^+$,$\mu-$. These pions then decay to produce muons according to:
\begin{equation}
    \pi^{+} \longrightarrow \mu^{+} + \nu_{\mu}
\end{equation}
\begin{equation}
    \pi^{-} \longrightarrow \mu^{-} + \Bar{\nu_{\mu}}
\end{equation}



While the decay of these primary particles produces mostly muons, Many other elementary particles are also produced
within this process. During the process of generation of primary particles that is the Kaons and pions, $K^0$ and $\pi^0$ are also produced
which on decay produces $\gamma$-particles. These Muons and other particles further decay to produce Electrons and Positrons.


\subsection{Properties of Muons}

Muons are unstable particles having intermediate mass between that of an 
electron and a proton, just like charged pions. Compared to pions, they are 
a little lighter.The muons carry one unit of electrical charge, either positive 
or negative, and are electrically charged. The life time of the muon is $2.2\mu s$.
The properties of the muons are tabulated below in Table-\ref{muonproperty}

\begin{table}[H]
    \centering
    \resizebox{0.85\columnwidth}{!}{%
    \begin{tabular}{|cc|c|}
        \hline
        \multicolumn{2}{|c|}{\textbf{Properties}}                           & \textbf{Values}              \\ \hline
        \multicolumn{1}{|c|}{\multirow{2}{*}{\textbf{Mass}}} & $m_{\mu}$    & $206.7686m_e$                \\ \cline{2-3} 
        \multicolumn{1}{|c|}{}                               & $m_{\mu}c^2$ & 105.659MeV                   \\ \hline
        \multicolumn{1}{|c|}{\textbf{Mean Life}}             & $\tau_{\mu}$ & $2.197 \mu s$                \\ \hline
        \multicolumn{1}{|c|}{\textbf{Spin}}                  & $s_{\mu}$    & 1/2                          \\ \hline
        \multicolumn{1}{|c|}{\textbf{Magnetic   Moment}}     & $\mu_{\mu}$  & $\frac{eh}{4\pi   m_{\mu} }$ \\ \hline
    \end{tabular}%
    }
    \caption{Properties of Muons}
    \label{muonproperty}
\end{table}

We know that the half life of muons are $1.56\mu s$. But we are able to observe 
the muons coming from outer space on the earth surface. This is due to the relativistic
effects. Now we know that the total energy of the cosmic muon is in the range of 4GeV.
Let us calculate the value of $\gamma$. Considering the relativity:
\begin{equation}
    \gamma m_{\mu} c^2 =4GeV = 6.4\times10^{-10} J
\end{equation}

Substituting $m_{\mu} = 1.883 \times 10^{-28} Kg$ we get:
\begin{equation}
    \gamma = 37.754770
\end{equation}


Now, let $t_{1/2}$ be the half life of muon when it is at rest WRT the earth and $t_{1/2}'$ 
be its half life when its moving. So, from time dilation, we have $t_{1/2}' =\gamma t_{1/2} $.
on substituting the values we get:
\begin{equation}
    t_{1/2}' \approx 0.058s
\end{equation}


As $\gamma$ is very high, the speed of the muons are very close to the speed of the
light. So time taken for the muons to reach the earth surface from the outer atmosphere is $\approx \text{distance}/c$
which is $\approx 10^{7}m/c \approx 0.03s$

So, this is why we are able to observe the cosmic muons incident on the earths surface
even if the half of the muons are in the range of micro seconds. The Calculations are based on an average scale.

%Cosmic Muons and its Properties%Cosmic Muons and its Properties%Cosmic Muons and its Properties%Cosmic Muons and its Properties%Cosmic Muons and its Properties




% Passage of Radiation Through Matter% Passage of Radiation Through Matter% Passage of Radiation Through Matter% Passage of Radiation Through Matter% Passage of Radiation Through Matter% Passage of Radiation Through Matter
\section{Passage of Radiation Through Matter}

Naturally, penetrating radiation views matter as a collection 
of electrons and nuclei along with their subatomic particles, 
which are its fundamental building blocks. Reactions with the 
atoms or nuclei as a whole, or with each of their individual 
constituents, may take place through any channels that are permitted,
depending on the type of radiation, its energy, and the type of 
material. The Coulomb force, electromagnetic collisions with atomic 
electrons, elastic scattering from a nucleus, absorption in a 
nuclear reaction, and other processes can all occur when an alpha 
particle enters a gold foil, for instance. These occur with a 
certain probability that is determined by the fundamental 
interactions that are involved, as well as by the laws of quantum 
mechanics. 



\subsection{The Cross section of the Interactions}

The cross section is a common way to describe how two particles 
collide or interact. If the fundamental interaction between the 
particles is known, this quantity can be calculated and serves as a 
gauge of the likelihood that a reaction will take place. According 
to formal definitions, the cross-section is defined as follows. 
Think about a particle beam that hits a target particle 2, as 
seen in Fig. 2.1. Assume that the target is much farther away 
from the beam and that the beam's particles are evenly spaced out 
in time. After that, we can talk about a flux of incident particles 
per unit of space and time. 

Now, considering the amount of particles scattering into an angle 
$d\Omega$ per unit time, the number of particles is not constant if we 
measure more number of time. Let the average no of particles be $N_s$ and let F
be the flux. The differential cross section is defined as :
\begin{equation}
    \frac{d\sigma}{d\Omega} (E,\Omega) = \frac{1}{F}  \frac{N_s}{d\Omega}
\end{equation}

that is, $d\sigma/d\Omega2$ is the average fraction of the particles scattered into df2 per unit time per
unit flux F. In terms of a single quantum mechanical particle, this may be reformulated
as the scattered probability current in the angle $d\Omega$ divided by the total incident
probability passing through a unit area in front of the target.


\begin{figure}[H]
    \centering	
     \includegraphics[width=\columnwidth]{crossection.png}
     \caption{CDiagram showing the cross section-\cite{leo1988techniques}}
     \label{CSdiag}
\end{figure}

So, the total cross section is given by:

\begin{equation}
    \sigma(E)=\int d \Omega \frac{d \sigma}{d \Omega}
\end{equation}


Assuming that the target centers are uniformly distributed and 
the slab is not too thick so that the likelihood of one center sitting 
in front of another is low, the number of centers per unit perpendicular 
area which will be seen by the beam is then $N\delta x$ where N is the 
density of centers and $\delta x$ is the thickness of the material along 
the direction of the beam. If the beam is broader than the target and A is 
the total perpendicular area of the target, the number of incident particles 
which are eligible for an interaction is then FA. The average number scattered 
into $d\Omega$ per unit time is then:

\begin{equation}
    N_s(\Omega) = FAN\delta x \frac{d\sigma}{d\Omega}
\end{equation}

and,

\begin{equation}
    N_{tot} = FAN \delta x \sigma
\end{equation}

And the probability of interaction in $\delta x$ is $N\sigma \delta x$-\cite{leo1988techniques}

\subsection{Interaction Probability}

Here we will calculate what id probability that a particle does not involve
in an interaction for a distance of x, this probability is known as the 
survival probability $P(x)$. Let the probability of having an interaction between 
$x$ and $dx$ be $wdx$. Then we have the probability of not having an interaction
between $x$ and $x+dx$ as:

\begin{equation}
    P(x+dx) = P(x)(1-wdx)
\end{equation}

On solving, we get the P(x) as :
\begin{equation}
    P(x) = C\exp{-wx}
\end{equation}
C turns out to be 1 while substituting the usual probability properties.


Now as we have the interaction probability, we will calculate the mean free
path, which is defined as:
\begin{equation}
    \lambda = \frac{\int x P(x)dx}{\int P(x)dx} = \frac{1}{w}
\end{equation}

but the interaction probability depends on the $\delta x$ intuitively, after 
approximating to the linear order terms we get:
\begin{equation}
    \lambda = \frac{1}{N\sigma}
\end{equation}


So the survival probability becomes:

\begin{equation}
    P(x) = \exp{(\frac{-x}{\sigma})}=\exp{(-N\sigma x)}
\end{equation}


\subsection{Energy Loss of the penetrating particle by Atomic Collisions}

Inelastic collisions with atomic electrons and the elastic scattering from the nuclei
are the two main reasons for the energy loss and change in direction of the particle.

Of course, the inelastic collisions are statistical in nature and have a 
certain quantum mechanical probability of happening. The fluctuations in 
the total energy loss are, however, small due to their abundance per 
macroscopic path length, so one can effectively work with the average 
energy loss per unit path length. Bohr first calculated this quantity 
often referred to as the stopping power or simply dE/dx—using classical 
reasoning. Later, Bethe, Bloch, and others did so using quantum mechanics.


\subsubsection{Bohr's Calculation}

Consider a heavy particle traveling through a material medium with a 
charge ze, mass M, and velocity v. Assume that an atomic electron is 
present at a distance and from the particle trajectory as shown in 
Figure-\ref{bohrsct1} 
In order to capture the electric field acting 
on the electron at its initial position, we assume that the electron 
is free, initially at rest, and that it only moves very slightly during 
the interaction with the heavy particle. Furthermore, because of its much 
greater mass (M>m), we assume that the incident particle will have 
essentially maintained its original course after the collision. 
This is one justification for separating heavy particles from electrons!




\begin{figure}[H]
    \centering	
     \includegraphics[width=\columnwidth]{bohrsct.png}
     \caption{Diagram For Bohr Calculation of Scattering-\cite{leo1988techniques}}
     \label{bohrsct1}
\end{figure}


Now, we will calculate the energy gained by the electron by finding the momentum impulse
it receives by colliding with the heavy particle.so:
\begin{equation}
    I = \int F dt = e\int E_{\perp} dt =e\int E_{\perp} \frac{dx}{v}
\end{equation}



By applying the gauss law, we get,

\begin{equation}
    \int E_{\perp} 2\pi bdx = 4\pi ze 
\end{equation}

So that we have:

\begin{equation}
    I = \frac{2ze^2}{bv}
\end{equation}

The energy gained by the electron is then:
\begin{equation}
    \label{eq1}
    \Delta E(b) = \frac{I^2}{2m_e} \frac{2z^2e^4}{mev^2b^2}
\end{equation}

Now, let the density of particles be $N_e$, then the energy lost to all the electrons
in the range b to b+db is given by:
\begin{equation}
    -dE(b) = \Delta E(B) N_e dV = \frac{4\pi z^2e^4}{mev^2} N_e \frac{db}{b} dx
\end{equation}

On solving with volume element $dV = 2\pi b db dx$ we get:

\begin{equation}
    -\frac{db}{dx} = \frac{4\pi z^2 e^4}{m_e v^2}N_e \ln{(\frac{b_{max}}{b_{min}})}
    \label{bmaxbmin}
\end{equation}

Ideally the limit should be 0 to infinity, but due to our assumption that
the collision at large b wont take place over a short period of time, there is an upper bound
$b_{max}$ and also at b=0, the integral diverges so there is also a $b_{min}$.

To calculate the $b_{min}$, the maximum kinetic energy is transferred when there is
a head on collision and the maximum energy it can gain is $\frac{1}{2}m_e (2v)^2$,
taking relativity we get this energy as $2\gamma^2 m_e v^2 $
substituting this to Equation-\ref{eq1}, we get,
\begin{equation}
    b_{min} = \frac{ze^2}{\gamma m_e v^2}
\end{equation}



Now for the calculation of $b_{max}$, We should look at the electrons. These electrons are not Free but bound to an atoms with some
orbital frequency $\nu$. For the electron to absorb some energy, the perturbation caused by the incident particle 
should be for a short time as compared to the angular frequency (1/$\nu$). Otherwise the perturbation 
will be adiabatic and there will be no transfer od energy. For Collisions, the interaction time is $t=b/v$. Considering the
relativistic effects, t becomes $t/\gamma$. so we can write:
\begin{equation}
    \frac{b}{\gamma v} \le \tau = \frac{1}{\bar{\nu}}
\end{equation}
where $\bar{nu}$ is the mean frequency averaged over all the states. The the upper limit of b becomes:
\begin{equation}
    b_{max} = \frac{\gamma v}{\bar{\nu}}
\end{equation}

Now substituting this into the Equation-\ref{bmaxbmin}, we get the classical bohr formula for the Energy Loss as:-\cite{leo1988techniques}

\begin{equation}
    -\frac{db}{dx} = \frac{4\pi z^2 e^4}{m_e v^2}N_e \ln{(\frac{\gamma^2 m_e v^3}{z e^2 \bar{\nu}})}
    \label{classicalbohr}
\end{equation}



\subsubsection{The Bethe-Bloch Formula}


The more realistic Quantum mechanical formulation of the Energy loss is carried out by
Bethe and Bloch in which the Energy is parametrized in terms of momentum rather than
the impact parameter. The formula thus obtained is:

\begin{equation}
    -\frac{db}{dx} = 2 \pi N_a r_e^2 C^2 \rho \frac{z^2 Z}{A \beta^2} (\ln{(\frac{2 m_e \gamma^2 v^2 W_{max}}{I^2})}-2\beta^2)
\end{equation}
Where $r_e$ is the electron radius, $N_a$ is the Avogadro Number, $\rho$ is the density of the material,
$m_e$ is the electron mass, I is the mean excitation potential, Z and A are the atomic
number and atomic mass of the absorbing material, z is the charge of the incident particle in units of e,
$W_{max}$ is the maximum energy transfer in a single collision, $\beta$ and $\gamma$ have the usual
definition in terms of velocity of the incident particle.

The maximum Energy Transfer id given by the formula:
\begin{equation}
    W_{max} = \frac{2m_e c^2 \eta^2}{1+2s\sqrt{1+\eta^2}+s^2}
\end{equation}
where $s=m_e/M$, where M is the mass of the incident particle and $\eta=\beta\gamma$.

Normally, two corrections are also added to the bethe bloch formula which is then given by:



\begin{eqnarray}
        -\frac{db}{dx} = 2 \pi N_a r_e^2 C^2 \rho \times\frac{z^2 Z}{A \beta^2} \times (\ln{(\frac{2 m_e \gamma^2 v^2 W_{max}}{I^2})} \\ -2\beta^2 - \delta - 2\frac{C}{Z})
\end{eqnarray} 


Where C is the Shell Correction and $\delta$ is the density Correction.


% Passage of Radiation Through Matter% Passage of Radiation Through Matter% Passage of Radiation Through Matter% Passage of Radiation Through Matter% Passage of Radiation Through Matter% Passage of Radiation Through Matter



%Simulation Using Geant4%Simulation Using Geant4%Simulation Using Geant4%Simulation Using Geant4%Simulation Using Geant4%Simulation Using Geant4

\section{Simulation Using Geant4}


Geant4-\cite{agostinelli2003geant4}, short for "Geometry and Tracking 4," is a powerful and 
widely-used toolkit for simulating the interactions of particles with matter. 
Developed by the CERN collaboration and used extensively in high-energy physics, 
nuclear physics, and other scientific fields, Geant4 is renowned for its accuracy 
and versatility. The toolkit provides a comprehensive set of tools and 
libraries that allow researchers to design complex geometries, define 
various particle sources, and study the interactions of particles with matter 
in intricate detail. Its modular and extensible architecture makes it adaptable 
to a wide range of simulation tasks, from studying the behavior of elementary 
particles in particle accelerators to modeling radiation therapy treatments in 
medical physics.


\subsection{The Detector Geometry}

The Detector Geometry is defined almost same as that is present in the laboratory to detect the muons. The Detector
construction is implemented using G4box, G4Tubs and G4Polyhedra. The logical volumes are crated using the G4LogicalVolume
and it is placed in appropriate positions just like what we have in the lab.

The detector that we are using in the lab is a Cosmic flux Photon Multiplicity Detector (CofPMD) in which we will be detecting
the Muons using the $Ar+CO_2$ (9:1) mixture present in the detector. So We will be setting the Sensitive detector as the Logical volume
of the Gas mixture. Using this Sensitive Detector, we will Measure the position of incidence and the Total  Energy deposition
from the Sensitive Detector.

\subsubsection{The Honeycomb shaped Detector}

Our main part of the detector is a honeycomb shaped detector made up of Copper in which at the Centre of each
hexagon there is a thin gold wire and this acts as the Anode in the real case and the Cu part acts as the Cathode. This whole
honeycomb is filled with a Mixture of Argon and Carbon Dioxide in the ratio 9:1. The Inner Radius of the Hexagon is $2.5mm$
and its depth is $5mm$. The radius of the Gold wire present at the centre is 10 microns. The construction of the honeycomb
and the gold wire is done using G4Polyhedra and G4Tubs respectively. The Constructed Hexagonal Honeycomb Structure is 
shown in the Figure-\ref{honeycomb}.

\begin{figure}[H]
    \centering	
     \includegraphics[width=\columnwidth]{honeycomb.png}
     \caption{Honeycomb structure created from GEANT4}
     \label{honeycomb}
\end{figure}

\begin{figure}[H]
    \centering	
     \includegraphics[width=\columnwidth]{honeycomb1.png}
     \caption{Honeycomb structure and Scintillation detectors created from GEANT4 (The RED BLUE and GREEN Lines are x,z and y axes)}
     \label{honeycomb1}
\end{figure}
The detector also contains Two scintillation detectors (of Thickness 1 cm and length of 11cm)  which are placed one above and one below at equal distances of 11cm
from the honeycomb detector. This is shown in the Figure-\ref{honeycomb1}







\subsubsection{The Complete Setup}

The complete Setup of the Experiment is implemented using the Detector
construction in GEANT4. The Complete construction is Shown in the 
Figure-\ref{combsetup}. The Grey parts of the simulations are made up
of Aluminum and the green parts are made up of plastic. Also the Electronics and
the PMDs are constructed whose material is FR4.

\begin{figure}[H]
    \centering	
     \includegraphics[width=\columnwidth]{combsetup.png}
     \caption{The complete Experimental Setup Simulation}
     \label{combsetup}
\end{figure}





\subsection{Generation of Particles}

The particle generation and its projection is done using the
usual fParticleGun with the G4VUserPrimaryGeneratorAction in Geant4.
Using this one can generate as many particles as we want in a single run.
for our case we will be generating muons only for the single particle case and
and for the Actual simulation of the Cosmic Shower, we will be using the
EcoMug for the data for different types of particles.




\subsection{Collection of Energy Deposition Data}
The energy Deposit form the Gas Mixture is calculated by the 
sensitive detector which was assigned for the Argon Gas mixture.
Geant4 Calculates the Energy Deposits and then it writes the values
for each run (each run consist of shooting of a single muon perpendicular to the 
Gas detector) into a text file. Which is then read by the Python and ROOT for further analysis.
The implementation  of the Detection and its calculation is 
done using G4VSensitiveDetector and G4Track.


%Simulation Using Geant4%Simulation Using Geant4%Simulation Using Geant4%Simulation Using Geant4%Simulation Using Geant4%Simulation Using Geant4




%Single Muon Interaction%Single Muon Interaction%Single Muon Interaction%Single Muon Interaction%Single Muon Interaction%Single Muon Interaction



\section{Single Muon only Interaction}

In this case, we will be using the fParticleGun to Generate $\mu^-$-Particles 
and will project them to the Detector through the scintillation detector 
parallel to the Z-Axis. We will be defining the momentum in terms of Energy (MeV and GeV) and all of this
with Construction of the detectors as mentioned above. 


\subsection{Obtained Energy Deposit data}

As mentioned earlier, the $\mu^-$-particles are fired at different
energies. Then its Most probable Energy Deposit Values are taken 
and plot against the incident energy and the graph is Checked by the
bethe-bloch formula.


\begin{figure}[H]
    \centering	
     \includegraphics[width=\columnwidth]{ED4GeVaccu.png}
     \caption{Accumulated paths of the muons for 500 Runs-\cite{agostinelli2003geant4}}
     \label{ED4GeVaccu}
\end{figure}


\begin{figure}[H]
    \centering	
     \includegraphics[width=\columnwidth]{ED4GeV.png}
     \caption{Statistics of Energy Deposit by 4GeV Muon Minus-\cite{ROOT}}
     \label{ED4GeV}
\end{figure}




\begin{figure}[H]
    \centering	
     \includegraphics[width=\columnwidth]{betheplot.png}
     \caption{Variation of MPV of Energy Deposit with the incident Energy of Muons-\cite{python}}
     \label{betheplot}
\end{figure}


\subsubsection{Single Energy Muon}

At first, we fired one particle at 4GeV perpendicular
to the detector through the scintillator in one run (100000 Runs were made). The plot of the range of
Energy Deposit is Shown in the Figure-\ref{ED4GeV} where we can see
that the distribution is landau. The function is Fitted by the
Landau function and MPV of the Energy Deposit is obtained as 677.82eV and the standard deviation
is 136.66eV. The corresponding image of the accumulated paths of the muons
as shown in the Figure-\ref{ED4GeVaccu}










\subsubsection{Muons With Different Energies}

Now, we will be calculating the Most Probable Value (MPV) of the
Energy Deposit for different values of the incident energy. We will Start 
from low energy (100MeV) and will end at calculating MPV of each Energy deposition
till we reach 100GeV particles. For Each energy we will be firing 100000 $\mu^-$-particles
one by one for each run. The energy Deposit is Calculated by the Sensitive Detector 
(Argon+$CO_2$ Mixture). 




The histogram showing the distributions for different values of the incident
energy of the Muons is Shown in the Figure-\ref{multihist}. From these Plots, The histograms 
are fitted with the Landau Function and the corresponding MPV of the Energy Deposits are taken out.
It is then plotted against the $log_10 ()$Value of incident energy as Shown in the Figure-\ref{betheplot}
Also, in the plot, we have made the comparison with the total
energy deposit value obtained from the Bethe Bloch Formula







\begin{figure*}[ht]
    \centering	
     \includegraphics[width=1.9\columnwidth]{multihist.png}
     \caption{Histograms Showing the Distribution of Energy Deposited by Muons incident at different Energies -\cite{ROOT}}
     \label{multihist}
\end{figure*}


%Single Muon Interaction%Single Muon Interaction%Single Muon Interaction%Single Muon Interaction%Single Muon Interaction%Single Muon Interaction




\section{Simulation of Actual Cosmic Muons}

In this Section, we will do the simulation of the actual cosmic muons that is generated
by the process mentioned in the above section-\ref{muonproduction}. In order to 
simulate the actual cosmic muons. We need the initial position and momentum of the
muons. For our case this wont be done in Geant4, but this is carried out in a 
single standalone header file called EcoMug.


\subsection{Cosmic Muon properties From EcoMug}

For the actual simulation of the cosmic muons, we require the initial position,
magnitude and direction of the momentum and the charge of the particle (to determine if 
its a muon minus or muon plus particle). The EcoMug is first set use the
simulation in the Flat-Sky mode with the center of the sky at the top
most point of the world volume of the geant4 (ini our case its (0,0,5m)).
The size of the Flat Sky is set to be $10m \times 10m $. Around 2.2 million
cosmic muon data was produced using the monte-carlo techniques used in the 
EcoMug. The data is generated in such a way that the only the relevant 
data is generated to use for the geant4. The the value of the Zenith angle of
the direction of momentum is restricted to the range of $2.49801-\pi$ Radians.
The angle is found out from the actual Experimental setup so that only the muons
that pass through both the scintillation detectors and the CoFPMD is detected. 




\subsubsection{The Zenith angle (\texorpdfstring{$\theta$}{theta}) distribution of Cosmic Muons}

From the several lakhs of Cosmic Muons Generated, a histogram of the
distribution of the zenith angle ($\theta$) of muon momentum direction is made and it is shown in the
Figure-\ref{zenith}




\subsubsection{The Azimuthal Angle Distribution}

Just like the Zenith angle, the Azimuthal angle $\phi$ distribution
from the Eco Mug is made and is shown in the Figure-\ref{azimuthal}.
From the Figure we can see the average Azimuthal angle is $\pi$



\subsubsection{Momentum Distribution of Muons}

Similarly, the distribution of the cosmic muon momentum defined in terms of its
Energy/c is made and is shown in the Figure-\ref{momentum}. It
is observed that the average energy of the incident muons on the earth
surface is 3GeV.





\begin{figure}[H]
    \centering	
     \includegraphics[width=\columnwidth]{zenith.png}
     \caption{Zenith Angle $(\theta)$ Distribution of the Cosmic Muon Momentum -\cite{ROOT}\cite{python}}
     \label{zenith}
\end{figure}


\begin{figure}[H]
    \centering	
     \includegraphics[width=\columnwidth]{azimuthual.png}
     \caption{Azimuthal Angle $(\phi)$ Distribution of the Cosmic Muon Momentum-\cite{ROOT}\cite{python}}
     \label{azimuthal}
\end{figure}

\begin{figure}[H]
    \centering	
     \includegraphics[width=\columnwidth]{momentum.png}
     \caption{Momentum Distribution of the Cosmic Muon Momentum-\cite{ROOT}\cite{python}}
     \label{momentum}
\end{figure}





\subsection{Cosmic Muons in Geant4}

The data from the EcoMug is written into a text file from which the it is converted to
a macro-file that can be read by the Geant4 by a python script.
The macro file then run by the geant4 using the previous geometry of detectors
and the corresponding energy deposit is calculated.


\subsubsection{The Distribution of Energy Deposit}


The distribution of the Energy deposit observed in the Geant4
analyzed in ROOT-\cite{ROOT} is Shown in the Figure-\ref{cosmu}.

\begin{figure}[H]
    \centering	
     \includegraphics[width=\columnwidth]{cosmu.png}
     \caption{MDistribution of Energy Deposit by Cosmic Muons-\cite{ROOT}\cite{python}}
     \label{cosmu}
\end{figure}


In the plot, Log scale is used and the particle flux is plotted against different energy Deposit.






\section{Conclusion}


In conclusion, this internship has been a remarkable journey into the realm of cosmic muon detection using the CoFPMD (Cosmic Flux Photon Multiplicity Detector). The multifaceted nature of the internship, ranging from constructing the detector setup to simulating real-world data, has yielded invaluable insights into the behavior of cosmic muons and the performance of the CoFPMD detector.

One of the primary objectives was to bridge the gap between simulation and reality, and the results have been promising. The alignment between observed data and simulated outcomes, with a close approximation of 1 muon per $cm^2$ per minute, underscores the accuracy and reliability of the CoFPMD detector's capabilities. This alignment not only validates the efficacy of the simulation methodology but also highlights the potential of the CoFPMD detector in accurately capturing and quantifying cosmic muons.

The investigation into the energy deposit by muons at various incidence energies has provided a deeper understanding of the interaction between cosmic muons and the Argon-CO2 gas mixture within the detector. The resulting energy deposit plots have showcased distinct patterns, shedding light on the intricate nature of cosmic muon behavior and their traversal through the detector materials.

The incorporation of actual data from EcoMug into the simulation has significantly enriched the authenticity of the results. This endeavor reflects the integration of theoretical concepts with real-world observations, bolstering the confidence in the simulation outcomes and enhancing the practical relevance of the study.













\end{multicols}
\bibliographystyle{plain}
\bibliography{bib.bib}


\end{document}