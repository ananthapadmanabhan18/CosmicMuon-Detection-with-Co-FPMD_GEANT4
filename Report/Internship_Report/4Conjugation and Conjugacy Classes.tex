\section{Conjugation and Conjugacy Classes}
We saw the action of G on a set m. Now lets consider the action of G on itself. When G acts on itself by left multiplication, a  transforms b into ab, this transformation is always transitive.


The action of G on itself is called a conjugation if the action takes "b" to $aba^{-1}$. Conjugation always defines a group action, since the action of ac on b yeilds . 
\begin{equation}
    (ac)b(ac)^{-1}=a(cbC^{-1})a^{-1}
\end{equation}
which is same as the action of c followed by action of a on b.
 Now the orbits of these groups under conjugation are called \textit{Conjugacy classes} of the group. The condition for 2 elements to belong to the same group is that, there exist some a such that:
 \begin{equation}
     aba^{-1}=c
 \end{equation}
from this, we can make 2 conclusions:
\begin{enumerate}
    \item The identity is always a one element conjugacy class as $aea^{-1}=a$ for any a.
    \item For an Abelian group, every element is in a conjugation class by itself as $aba^{-1}=aa^{-1}b=b$ since G is an Abelian group.
\end{enumerate}
